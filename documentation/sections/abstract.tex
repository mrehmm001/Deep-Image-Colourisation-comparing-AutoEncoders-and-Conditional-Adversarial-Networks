\begin{abstract}
\addchaptertocentry{\abstractname} % Add the abstract to the table of contents
The process of image colorization involves taking a grayscale image and adding color information to it in a way that looks realistic. Early methods were used to rely on user-guided techniques such as scribbling and using reference images. But nowadays, deep learning techniques for image colourisation has been introduce, allowing the process to be fully automated which eradicates the need for user-guidance. There are several types of deep learning architectures that can achieve image colourisation such as AutoEncoders and conditional adversarial networks. This investigation only focuses on deep learning techniques for image colourisation, specifically AutoEncoders and cGAN. The invastigation aims compare the performance between AutoEncoders and cGAN using quantatative metrics as well as highlighting limitations seen through each method. In addition to this, the investigation also looks into whether objective metrics such as PSNR and SSIM correlate with human opinion in regards to image colourisation and determine which method of evaluation is best suited when analysing image colourisation methods. Finally, we discuss limitations suffered from each method and go over possible research as future work.

\textbf{Keywords - \keywordnames}
\end{abstract}

