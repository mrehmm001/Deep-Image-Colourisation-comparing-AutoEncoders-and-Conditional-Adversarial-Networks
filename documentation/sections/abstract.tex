\begin{abstract}
\addchaptertocentry{\abstractname} % Add the abstract to the table of contents

Image colorization is the process of taking a grayscale image and adding color information to it in a way that looks realistic. Early methods for colorization relied on user-guided techniques, such as scribbling and using reference images. However, nowadays deep learning techniques for image colorization have been introduced, which eradicates the need for user guidance. There are several types of deep learning architectures that can achieve image colorization, such as AutoEncoders and conditional adversarial networks. This investigation only focuses on deep learning techniques for image colorization, specifically AutoEncoders and cGANs. The investigation aims to compare the performance between AutoEncoders and cGANs using quantitative metrics. In addition to this, the investigation also looks into whether objective metrics, such as PSNR and SSIM, correlate with human opinion in regard to image colorization and determines which method of evaluation is best suited when analyzing image colorization methods. The limitations suffered from each method and possible future research are also mentioned. The investigation concludes that using objective metrics for evaluation is unreliable for measuring colors from the image and using user studies is the optimal solution. The cGAN models performed overall the best, by producing more realistic colorization and adding a higher variety of colors.

\textbf{Keywords - \keywordnames}
\end{abstract}

