
\chapter{Introduction}
\pagenumbering{arabic}
\section{Purpose and Motivation}
The main \textbf{purpose} of this report is to investigate various techniques used for the application of image colourisation and restoration purposes using deep learning tactics such as AutoEncoders and Conditional Generative Adversarial Network (cGAN). These purposes may include applying colour to an old black and white image or restoring an old coloured image to its original form as accurately as possible.  

The \textbf{motivation} for my project came from watching historical documentaries that were recently colourised using artificial intelligence. The colourisation added a lot more sense of immersion; it made every human being, every car, every horse, and everything look more real and alive. What was more impressive was that it was all done by AI. This fascinates me because, for the human mind, the idea of image colourisation is a simple task. We learn this from a very early age—to fill in the missing colours from a colouring book, by knowing the fact that the sky is blue, the grass is green, clouds are white, apples are red or green. Using deep learning, we can replicate that mindset for a machine to automate the process of colourising photos for us.

On top of that, I also thought that having such a model would allow it to be a useful tool for users to apply colours to old black and white image artefacts or restore videos and images that have suffered deterioration in quality over time. 

Moreover, AI image colourisation can be used to reveal semantic meaning or details that may have otherwise been left out or lost from the image. Besides that, having an automated colourisation process can eliminate the need to hire an artist (or can be used to assist them) to colourise or restore your photos, as such a model will be able to do what an image restoration artist can, for free and fast.

\section{Aims}
The \textbf{aim} of this project is to determine the best image colourisation method by \textbf{carrying out analysis} of \textbf{deep learning architectures such as AutoEncoders and Conditional Adversarial Networks} using \textbf{quantitative measures such as human assessment and objective metrics}.

In addition to this aim, the project will investigate whether the \textbf{objective measures correlates with human assessment} results and \textbf{determine which measure matters more for the purpose of image colourisation}.

\section{Scope}
The \textbf{scope} of this research project is to conduct an analysis and evaluation of architectures such as AutoEncoders and cGAN for the task of image colourisation, and determine which method is best suited for this task. A total of four existing method architectures will be used for this study. The first two methods are AutoEncoders whereas the second two are cGANs. All methods will be trained using a dataset, Places365 which is split into 60:20:20 for training, testing and validation respectively. PSNR and Naturalness perception study will be the metrics used to evaluate each method's performance in the test set. Visualisations will also be used to showcase the colourisation ability.  


\section{Section Overview}
\textbf{Chapter 2: Background Research}: This chapter goes over the intuition about the history and process of image colourisation, as well as the area of research within deep learning and similar projects.

\textbf{Chapter 3: Methodology} This section of the report will go over the step by step methodology used to conduct my research, including the data preprocessing. Here, I will be describing the methods I have used to experiment with image colourisation.

\textbf{Chapter 3 Research results:} In this chapter, I will be going through the results gathered from the implementation and training. Each method will be tested against unseen test data. Their results will be in the form of visualisations and quantitative metrics used to evaluate each method's performance and determine which class method is best suited for the purpose of image colourisation. In addition to that, I will discuss the results along the way and go over any existing limitations suffered from each method.

\textbf{Chapter 4 Conclusion & future work:} The final chapter of the report will conclude the investigation by going over what has been accomplished along the way but also possible improvements which may yield better results for this investigation as future work.




% \section{Objectives}
% In order to achieve my aims, I will be following a thorough deep learning workflow methodology, where I will iteratively develop and tune and test models until I derive an optimal model. Python will be used to develop the model using the TensorFlow deep learning framework.
